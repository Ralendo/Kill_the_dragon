\begin{frame}
    \titlepage
\end{frame}


\begin{frame}{Исходник}
\begin{itemize}
    \item \href{https://github.com/Ralendo/Kill_the_dragon}{GitHub}
\end{itemize}
\end{frame}

\begin{frame}{Формулировка задачи}
\begin{itemize}
    \item Есть маги, которые делятся на лекарей и атакующих магов.
    \item Когда маг вне безопасной точки, он теряет силу своей магии. А когда в ней, то наоборот, набирает. 
    \item Дан дракон, которого нужно победить.
    \item У дракона имеются атаки "Огненный Шторм", "Дыхание Дракона" и "Удар хвостом"
    \begin{itemize}
        \item "Огненный Шторм" - имеет одну общую безопасную точку для всех магов
        \item "Дыхание Дракона" - имеет безопасные точки, количество = магам.
        \item "Удар хвостом" - убивает всех магов.
    \end{itemize}
    \item Дракона можно убить только опустив его запас здоровья до 0 и только во время атаки "Огненный Шторм"
    \item Если хотя бы у одного из магов запас здоровья опустится до 0, то все маги умирают.
\end{itemize}
\end{frame}

\begin{frame}{Запись исходных данных}
\begin{itemize}
    \item Импорт библиотеки NumPy, для более удобной работы с массивами
    \item Вводим исходные данные до атак.
    \item Формируем два массива с аурой: один для лекарей, других для атакующих магов. Первый столбец - макс. мощность ауры, второй - скорость роста ауры, третий - скорость уменьшения ауры. 
    \item Формируем ещё два массива, где первый столбец - текущий запас здоровья, а второй - текущая аура.
\end{itemize}
\end{frame}

\begin{frame}{Структура программы}
\begin{itemize}
    \item Вся задача сводится к тому, чтобы обрабывать атаки Дракона.
    \item Во время атак, у магов есть два состояния: либо аура выключена и они бегут к точке, либо аура включена и они используют свои заклинание (лечение и нанесение урона). Эти состояния оформлены в качестве функций.
    \item В конце программы пишем блок с вводом типа атаки. Из него будут вызываться события "Атаки Дракона", которые ранее прописаны в качестве функций (процедур)
\end{itemize}
\end{frame}

\begin{frame}{Атака "Огненный Шторм"}
\begin{itemize}
    \item Помимо получения исходных данных, формируем для каждого типа мага по два массива: \begin{enumerate}
        \item Индексы маги вне безопасной точки.
        \item Индексы магов с не полной аурой.
    \end{enumerate}
    \item Проверка, имеется ли хоть один маг с не полной аурой и запуск итерации сразу по двум массивам для экономии времени алгоритма.
    \item В итерации, проверяем, действительно ли маг вне безопасной зоны, а затем, в зависимости от того, в безопасной ли он зоне или нет, вызываем функции состояний мага.
    \item После итерации, наносим урон дракону и лечим магов. Затем проверка, не умер ли кто-то и можем ли мы убить дракона.

\end{itemize}
\end{frame}

\begin{frame}{Атака "Дыхание Драко"}
\begin{itemize}
    \item Главная проблема: приоритет точек
    \item Доп. условия для сокращения времени программы.
    \item Алгоритм = Атака "Огненный Шторм"

\end{itemize}
\end{frame}

\begin{frame}{Алгоритм раздачи точек}
\begin{itemize}
    \item Выбираем точки в k - шаговой доступности.
    \item Если точка видна на радаре только для него, то ему.
    \item Иначе он идёт спрашивать у другого мага при помощи рекурсии.
    \item Умирающие -> Лекари -> Сильнейшие лекари.
\end{itemize}
\end{frame}

\begin{frame}{Что не учтено?}
\begin{itemize}
    \item Возможно такое, что маг не может добежать до 2-ух ходовой точки, но если он залечится на точке, которая находится в 1 ходу от него, то он сможет добежать и до 2-ух ходовой.
\end{itemize}
\end{frame}

\begin{frame}{Aura is online}
\begin{itemize}
    \item Повышаем мощность текущей ауры
    \item Обновляем общий хил и урон
\end{itemize}
\end{frame}

\begin{frame}{Aura is offline}
\begin{itemize}
    \item Передвигаем мага на его скорость
    \item Наносим урон по магу
    \item Понижаем мощность текущей ауры
\end{itemize}
\end{frame}

\begin{frame}{Источники}
\begin{itemize}
    \item \href{www.python.org}{Офиц. сайт Python}
    \item \href{https://acm.timus.ru/problem.aspx?space=1&num=1952}{Исходник задачи}
    \item \href{https://numpy.org/doc/}{Документация по библиотеке NumPy}
\end{itemize}
\end{frame}
